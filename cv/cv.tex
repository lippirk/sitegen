\documentclass[10pt]{article}
\usepackage{array, xcolor, lipsum, bibentry}
\usepackage[margin=3cm]{geometry}

\usepackage{hyperref}
% make href not ugly
\hypersetup{
    colorlinks=true,
    linkcolor=cyan,
    filecolor=magenta,
    urlcolor=black,
    }
\urlstyle{same}

\title{\bfseries Ben Anson}
\author{\href{mailto:benansondev@gmail.com}{benansondev@gmail.com}\\\url{https://github.com/lippirk}}
\date{}
\pagenumbering{gobble}
\setlength{\parindent}{0pt}

\usepackage{titling}
\setlength{\droptitle}{-1in}

\definecolor{lightgray}{gray}{0.8}
\newcolumntype{L}{>{\raggedleft}p{0.14\textwidth}}
\newcolumntype{R}{p{0.8\textwidth}}
\newcommand\VRule{\color{lightgray}\vrule width 0.5pt}

\begin{document}
\maketitle

I am a Master's student at the University of Edinburgh and former Software
Engineer. I am interested in using mathematics, statistics and software to
solve real world problems.

\subsection*{Education}
\begin{tabular}{L!{\VRule}R}
  2021-now&{{\bf University of Edinburgh}, \it MSc Statistics and Operational Research}\\
          &{\it Grade:} TBC\\
          &{\it Courses (studied so far) include:} Bayesian Theory, Statistical
          Programming, Fundamentals of OR / Optimization, Generalized
          Regression models.\\
2014-2017   &{{\bf University of Cambridge}, \it BA Mathematics}\\
            &{\it Grades:} Part IA: 2i, Part IB: 2i, Part II: 2i\\
            &{\it Courses include:} Analysis, Computational Projects, Linear Algebra, Methods, Probability.
I took mostly applied mathematics courses in my third year.\\
2007-2014   &{\bf Ermysted's Grammar School}\\
            &{\it A level:} A*A*A\\
            &{\it GCSE   :} 8A*s, 1A
\end{tabular}

\subsection*{Work experience}
\begin{tabular}{L!{\VRule}R}
2019-2021&{{\bf Citrix}, \it Software engineer}\\
         &Worked on a team that released the Citrix Hypervisor, a platform for
         running virtual workloads, which has users such as Red Bull Racing,
         Deutsche Bank and the NHS. Primarily contributed to the Xapi project,
         a large open source OCaml codebase which implements an API (the
         `XenAPI')         for managing distributed systems of Xen hypervisor
         hosts.  Led new product features.  Triaged and fixed bugs raised by
         Python test suite (XenRT). Packaged hotfixes. Buddied interns. Scrum
         master on a team of approximately 8 engineers.\\
2017-2019&{{\bf Anthony Best Dynamics}, \it Software developer}\\
         &Developed C\# applications used for testing vehicles against EuroNCAP
         and NHTSA safety protocols. Designed, wrote, and optimized a library
         to post-process data channels generated by vehicle driving robots, and
         worked on visualizing this data usefully for users.  Also worked on a
         live vehicle monitoring app.\\
\end{tabular}

\subsection*{Technical Skills by usage}
\begin{tabular}{L!{\VRule}R}
  &{\bf Citrix:} Production code mostly in {\it OCaml}; testing and automation in {\it Python, Bash}; also {\it Linux}, {\it Git}.\\
  &{\bf Anthony Best Dynamics:} Production code in {\it C\#}; also {\it Git, Visual Studio}.\\
  &{\bf Blog (\url{https://lippirk.github.io}):} Simulations, visualizations, static
          site generator in {\it JavaScript, TypeScript}; data wrangling in {\it
          Python}; POCs in {\it R}; also {\it HTML, CSS}.\\
  &{\bf Master's course:} Statistical modelling in {\it R}; solving MIP and LP problems with {\it Xpress, Gurobi}.\\
  &{\bf Undergraduate Computational Projects:} {\it MATLAB}, {\it Python}.\\
\end{tabular}

\end{document}
